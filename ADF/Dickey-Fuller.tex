\documentclass[11pt]{article}
\usepackage{enumerate,amsmath,amssymb,fancyhdr,mathrsfs,amsthm,url,yhmath}
\usepackage{setspace}
\usepackage{graphicx}
\usepackage{color}
\usepackage{tikz}
\onehalfspacing
\raggedbottom
\usepackage[myheadings]{fullpage}

\newcommand{\<}{\ensuremath{\langle}}
\renewcommand{\>}{\ensuremath{\rangle}}
\newcommand{\beps}{\ensuremath{\boldsymbol{\varepsilon}}}
\newcommand{\bA}{\ensuremath{\mathbf{A}}}
\newcommand{\id}{\ensuremath{\mathrm{id}}}
\newcommand{\bB}{\ensuremath{\mathbf{B}}}
\newcommand{\sS}{\ensuremath{\mathscr{S}}}
\newcommand{\Eq}{\ensuremath{\mathrm{Eq}}}
\newcommand{\Cg}{\ensuremath{\mathrm{Cg}}}
\newcommand{\Stab}{\ensuremath{\mathrm{Stab}}}
\newcommand{\Con}{\ensuremath{\mathrm{Con}}}
\newcommand{\Pol}{\ensuremath{\mathrm{Pol}}}
%\newcommand{\Pol1}{\ensuremath{\mathrm{Pol}_1}}
\newcommand{\ps}[1]{\ensuremath{^{(#1)}}}
\newcommand{\piB}{\ensuremath{\pi_B}}
\newcommand{\hpsi}{\ensuremath{\hat{\psi}}}
\newcommand{\htheta}{\ensuremath{\hat{\theta}}}
\newcommand{\supi}{\ensuremath{^{(i)}}}
\newcommand{\supj}{\ensuremath{^{(j)}}}
\renewcommand{\leq}{\ensuremath{\leqslant}}
\renewcommand{\nleq}{\ensuremath{\nleqslant}}
\renewcommand{\geq}{\ensuremath{\geqslant}}
\newcommand{\meet}{\ensuremath{\wedge}}
\newcommand{\join}{\ensuremath{\vee}}
\newcommand{\Meet}{\ensuremath{\bigwedge}}
\renewcommand{\Join}{\ensuremath{\bigvee}}

\pagestyle{fancy}
\lhead{{\it Notes on the Dickey-Fuller Test}}  \chead{} \rhead{September 24, 2011}

\begin{document}
{\renewcommand{\thefootnote}{}\footnotetext{William DeMeo \url{<williamdemeo@gmail.com>}}}
%\let\thefootnote\relax\footnotetext{\url{<williamdemeo@gmail.com>}}
\section{Background}
The Dickey-Fuller test is a standard method for testing whether or not a
time-series is stationary.  This note gives a brief description of what the test
does, and a bit about our implementation of it.

\subsection{Matlab's implementation}
The Matlab command {\bf adftest} provides an 
augmented Dickey-Fuller test for a unit root.\\
The models are: 
\begin{itemize}
\item 
AR (autoregressive)
\[
H_0: y_t = y_{t-1} + b_1 \Delta  y_{t-1} +  \cdots + b_p\Delta y_{t-p} + e_t.
\]
\[
H_1: y_{t} = a y_{t-1} + b_1\Delta y_{t-1} + \cdots + b_p\Delta y_{t-p} + e_t.
\]
with AR(1) coefficient $a < 1$.

\item ARD (autoregressive with drift)
tests AR null model against
\[
H_1:  y_{t} = c + a y_{t-1} + b_1\Delta y_{t-1} + b_2\Delta y_{t-2} + \cdots + b_p\Delta y_{t-p} + e_t.
\]
with drift coefficient $c$ and AR(1) coefficient $a < 1$.

\item TS (trend stationary) 
\[
H_0: y_{t} = c +  y_{t-1} + b_1\Delta y_{t-1} + \cdots + b_p\Delta y_{t-p} + e_t.
\]
\[
H_1: y_{t} = c + dt + a y_{t-1} + b_1\Delta y_{t-1} + \cdots + b_p\Delta y_{t-p} + e_t.
\]
with drift coefficient $c$, deterministic trend coefficient $d$, and AR(1) coefficient $a < 1$. 
\end{itemize}
The default model is AR.

We need to have options for at least three different
models, depending on our 
assumptions about the series in question, as these different models will lead to
different conclusions about stationarity.
For example, when I used the Matlab routine on a five macroeconomic time-series,
I found that, when the model is AR we could reject the null for all variables except inflation.
Whereas, when the model is ARD, we fail to reject the null for all variables except inflation.




\end{document}

